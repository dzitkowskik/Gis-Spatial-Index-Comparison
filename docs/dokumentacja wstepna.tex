\documentclass[11pt,a4paper]{article}

\usepackage[english,polish]{babel}
\usepackage[utf8]{inputenc}
\usepackage[T1]{fontenc}
\usepackage{lmodern}
\usepackage{indentfirst}
\usepackage{fullpage}
\usepackage{enumerate}
\usepackage{secdot}
\usepackage{verbatim}
\usepackage{listings}
\usepackage{graphicx}

\selectlanguage{polish}
\frenchspacing

% Zdefiniowanie autorów i~tytułu:
\author{Karol Dzitkowski}
\title{
	Aplikacje i usługi GIS\\
	\huge{Porównanie indeksów przestrzennych}\\
 	Dokumentacja wstępna
 } 	

\begin{document}
\maketitle
\newpage

% Wstawienie spisu tresci:
\tableofcontents
\newpage

%Treść:
\section{Cel projektu}
Porównanie różnego rodzaju indeksów przestrzennych i własnych pomysłów ze standardowo dostępnym indeksem przestrzennym w QGISie
(sprawdzenie wydajności wyszukiwania oraz odnajdywania najbliższych k punktów).
\section{Opis problemu}
Porównanie będzie polegało na przeprowadzeniu testów wydajności wyszukiwania, obliczania odległości, oraz innych operacji często
używanych w aplikacjach GIS, na dużym zbiorze danych. Porównywane będą indeksy przestrzenne oparte o:
\begin{enumerate}
	\item Brak indeksowania
	\item Functional Indexes np. lower(col1)
	\item GIN index
	\item B-Tree
	\item Geohashes
	\item Hilbert Curves
	\item GiST
	\item SP-GiST
\end{enumerate}
Wyniki przedstawiane będą w formie wykresów, a także w formie wygenerowanego raportu do pliku. Wyniki zostaną przeanalizowane i
po skończeniu projektu, zostanie utworzona dokumentacja opisująca wyniki i wyciągająca wnioski płynące z testów. Na tych samych danych
wejściowych zostanie przetestowana wydajność indeksowania dla systemu QGIS. Wyniki dla QGIS zostaną porównane do wyników wyżej opisanych 
testów w dokumentacji końcowej.
\section{Rozwiązanie}
\subsection{Program}
Rozwiązanie będzie oparte na technologii Microsoft $ASP.NET MVC4$ oraz będzie napisane głównie w języku $C\#$. Klient będzie zatem
systemem przeglądarkowym. Serwer oprogramowania będzie łączył się z bazą danych $PostgreSQL$ z wtyczką $PostGIS$ za pomocą biblioteki 
$Npgsql.dll$. Serwer w skutek zapytania od klienta, będzie odpalał odpowiednie procedury SQL i śledził czas ich wykonania, a następnie odsyłał wyniki. 
Program będzie umożliwiał porównywanie czasu wykonania różnych procedur/zapytań dla wszystkich, lub dwóch wybranych sposobów 
indeksowania. Czas wykonania w zależności od ilości danych do przetworzenia będzie prezentowany w formie wykresu. Wykresy 
generowane będą przy pomocy biblioteki $DotNet.Highcharts$ na licencji MIT.
\subsection{Dane}
Testy przeprowadzane będą na zbiorach danych o różnej strukturze oraz różnej wielkości. Dane brane pod uwagę:
\begin{itemize}
	\item Punkty wygenerowane losowo
	\item Linie (pary punktów) wygenerowane losowo
	\item Prawdziwe dane dla Polski zaczerpnięte z GeoCommunity (www.geocomm.com) w formacie E00.
\end{itemize}
Dane w formacie E00 zostaną przekonwertowane programem e00pg i zaczytane do bazy PostgreSQL. Reszta danych zostanie 
wygenerowana za pomocą procedur SQL bezpośrednio na bazie.
\subsection{Testy}
Analiza oraz porównanie różnych sposobów indeksowania będzie opierała się na badaniu wpływu indeksowania na wydajność dla różnych
zestawów danych o różnych charakterystykach i właściwościach. np. dla punktów losowych, obejmujących całą przestrzeń wartości i dla punktów skupionych
w pewnych obszarach. Testowanie wyżej wymienionych metod indeksowania odbędzie się tylko i wyłącznie z użyciem programu, natomiast testy dla
wydajności indeksowania QGIS odbędą się przy użyciu programu QGIS z tymi samymi danymi zaimportowanymi z bazy PostgreSQL. Wyniki testów
zostaną przedstawione w dokumentacji końcowej w postaci tabeli oraz wykresów.

\section{Źródła}
\begin{itemize}
	\item $https://www.cs.purdue.edu/spgist/publications.xml$
	\item $http://drum.lib.umd.edu/bitstream/1903/1335/2/CS-TR-4556.pdf$
	\item $http://blog.notdot.net/2009/11/Damn-Cool-Algorithms-Spatial-indexing-with-Quadtrees-and-Hilbert-Curves$
	\item $http://informix-spatial-technology.blogspot.com/2012/01/comparison-on-b-tree-r-tree-quad-tree.html$
	\item $https://www.youtube.com/watch?v=53onZyrn8vA$
	\item $http://www.postgresql.org/docs/8.4/interactive/indexam.html$
	\item $https://www.pgcon.org/2011/schedule/attachments/197_pgcon-2011.pdf$
	\item $http://stackoverflow.com/questions/22602722/which-geo-implementation-to-use-for-millions-of-points$
	\item $http://e00pg.sourceforge.net/$
	\item $http://spatialnews.geocomm.com/education/tutorials/e00data/$
	\item $http://data.geocomm.com/catalog/PL/datalist.html$
\end{itemize}

\end{document}