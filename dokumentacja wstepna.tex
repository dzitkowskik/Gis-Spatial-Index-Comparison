\documentclass[11pt,a4paper]{article}

\usepackage[english,polish]{babel}
\usepackage[utf8]{inputenc}
\usepackage[T1]{fontenc}
\usepackage{lmodern}
\usepackage{indentfirst}
\usepackage{fullpage}
\usepackage{enumerate}
\usepackage{secdot}
\usepackage{verbatim}
\usepackage{listings}
\usepackage{graphicx}

\selectlanguage{polish}
\frenchspacing

% Zdefiniowanie autorów i~tytułu:
\author{Karol Dzitkowski}
\title{
	Aplikacje i usługi GIS\\
	\huge{Porównanie indeksów przestrzennych}\\
 	Dokumentacja wstępna
 } 	

\begin{document}
\maketitle
\newpage

% Wstawienie spisu tresci:
\tableofcontents
\newpage

%Treść:
\section{Cel projektu}
Porównanie różnego rodzaju indeksów przestrzennych i własnych pomysłów ze standardowo dostępnym indeksem przestrzennym w QGISie
(sprawdzenie wydajności wyszukiwania oraz odnajdywania najbliższych k punktów).
\section{Opis problemu}
Porównanie będzie polegało na przeprowadzeniu testów wydajności wyszukiwania, obliczania odległości, oraz innych operacji często
używanych w aplikacjach GIS, na dużym zbiorze danych. Porównywane będą indeksy przestrzenne oparte o:
\begin{enumerate}
	\item Brak indeksowania
	\item Indeksowanie naiwne (najpierw X potem Y)
	\item B-Tree
	\item Quadtrees
	\item Geohashes
	\item Hilbert Curves
	\item R-Tree
\end{enumerate}
Wyniki przedstawiane będą w formie wykresów, a także w formie wygenerowanego raportu do pliku. Wyniki zostaną przeanalizowane i
po skończeniu projektu, zostanie utworzona dokumentacja opisująca wyniki i wyciągająca wnioski płynące z testów.

\section{Rozwiązanie}
\subsection{Program}
Rozwiązanie będzie oparte na technologii Microsoft $ASP.NET MVC4$ oraz będzie napisane głównie w języku $C\#$. Klient będzie zatem
systemem przeglądarkowym. Serwer oprogramowania będzie łączył się z bazą danych $PostgreSQL$ z wtyczką $PostGIS$ za pomocą biblioteki 
$Npgsql.dll$. Serwer w skutek zapytania od klienta, będzie odpalał odpowiednie procedury SQL i śledził czas ich wykonania, a następnie odsyłał wyniki. 
Program będzie umożliwiał porównywanie czasu wykonania różnych procedur/zapytań dla wszystkich, lub dwóch wybranych sposobów 
indeksowania. Czas wykonania w zależności od ilości danych do przetworzenia będzie prezentowany w formie wykresu. Wykresy 
generowane będą przy pomocy biblioteki $DotNet.Highcharts$ na licencji MIT.

\subsection{Baza i indeksy}

\end{document}